\documentclass{amsart}

%\usepackage{etoolbox}
%\makeatletter
%\let\ams@starttoc\@starttoc
%\makeatother
%\makeatletter
%\let\@starttoc\ams@starttoc
%\patchcmd{\@starttoc}{\makeatletter}{\makeatletter\parskip\z@}{}{}
%\makeatother

%\usepackage[parfill]{parskip}

\usepackage[colorlinks=true,linkcolor=blue,citecolor=blue,urlcolor=blue]{hyperref}
\usepackage{bookmark}
\usepackage{amsthm,thmtools,amssymb,amsmath,amscd}

\usepackage[bibstyle=alphabetic,citestyle=alphabetic,backend=bibtex]{biblatex}
\bibliography{Bibliography}

\usepackage{fancyhdr}
\usepackage{esint}

\usepackage{enumerate}

\usepackage{pictexwd,dcpic}

\usepackage{graphicx}

\swapnumbers
\declaretheorem[name=Theorem,numberwithin=section]{thm}
\declaretheorem[name=Remark,style=remark,sibling=thm]{rem}
\declaretheorem[name=Lemma,sibling=thm]{lemma}
\declaretheorem[name=Proposition,sibling=thm]{prop}
\declaretheorem[name=Definition,style=definition,sibling=thm]{defn}
\declaretheorem[name=Corollary,sibling=thm]{cor}
\declaretheorem[name=Assumption,style=remark,sibling=thm]{ass}
\declaretheorem[name=Example,style=remark,sibling=thm]{example}


\numberwithin{equation}{section}

\usepackage{cleveref}
\crefname{lemma}{Lemma}{Lemmata}
\crefname{prop}{Proposition}{Propositions}
\crefname{thm}{Theorem}{Theorems}
\crefname{cor}{Corollary}{Corollaries}
\crefname{defn}{Definition}{Definitions}
\crefname{example}{Example}{Examples}
\crefname{rem}{Remark}{Remarks}
\crefname{ass}{Assumption}{Assumptions}
\crefname{not}{Notation}{Notation}

%Symbols
\renewcommand{\~}{\tilde}
\renewcommand{\-}{\bar}
\newcommand{\bs}{\backslash}
\newcommand{\cn}{\colon}
\newcommand{\sub}{\subset}

\newcommand{\N}{\mathbb{N}}
\newcommand{\R}{\mathbb{R}}
\newcommand{\Z}{\mathbb{Z}}
\renewcommand{\S}{\mathbb{S}}
\renewcommand{\H}{\mathbb{H}}
\newcommand{\C}{\mathbb{C}}
\newcommand{\K}{\mathbb{K}}
\newcommand{\Di}{\mathbb{D}}
\newcommand{\B}{\mathbb{B}}
\newcommand{\8}{\infty}

%Greek letters
\renewcommand{\a}{\alpha}
\renewcommand{\b}{\beta}
\newcommand{\g}{\gamma}
\renewcommand{\d}{\delta}
\newcommand{\e}{\epsilon}
\renewcommand{\k}{\kappa}
\renewcommand{\l}{\lambda}
\renewcommand{\o}{\omega}
\renewcommand{\t}{\theta}
\newcommand{\s}{\sigma}
\newcommand{\p}{\varphi}
\newcommand{\z}{\zeta}
\newcommand{\vt}{\vartheta}
\renewcommand{\O}{\Omega}
\newcommand{\D}{\Delta}
\newcommand{\G}{\Gamma}
\newcommand{\T}{\Theta}
\renewcommand{\L}{\Lambda}

%Mathcal Letters
\newcommand{\cL}{\mathcal{L}}
\newcommand{\cT}{\mathcal{T}}
\newcommand{\cA}{\mathcal{A}}
\newcommand{\cW}{\mathcal{W}}

%Mathematical operators
\newcommand{\INT}{\int_{\O}}
\newcommand{\DINT}{\int_{\d\O}}
\newcommand{\Int}{\int_{-\infty}^{\infty}}
\newcommand{\del}{\partial}

\newcommand{\inpr}[2]{\left\langle #1,#2 \right\rangle}
\newcommand{\fr}[2]{\frac{#1}{#2}}
\newcommand{\x}{\times}
\DeclareMathOperator{\Tr}{Tr}

\DeclareMathOperator{\dive}{div}
\DeclareMathOperator{\id}{id}
\DeclareMathOperator{\pr}{pr}
\DeclareMathOperator{\Diff}{Diff}
\DeclareMathOperator{\supp}{supp}
\DeclareMathOperator{\graph}{graph}
\DeclareMathOperator{\osc}{osc}
\DeclareMathOperator{\const}{const}
\DeclareMathOperator{\dist}{dist}
\DeclareMathOperator{\loc}{loc}
\DeclareMathOperator{\grad}{grad}
\DeclareMathOperator{\ric}{Ric}
\DeclareMathOperator{\Rm}{Rm}
\DeclareMathOperator{\weingarten}{\mathcal{W}}
\DeclareMathOperator{\inj}{inj}

%Environments
\newcommand{\Theo}[3]{\begin{#1}\label{#2} #3 \end{#1}}
\newcommand{\pf}[1]{\begin{proof} #1 \end{proof}}
\newcommand{\eq}[1]{\begin{equation}\begin{alignedat}{2} #1 \end{alignedat}\end{equation}}
\newcommand{\IntEq}[4]{#1&#2#3	 &\quad &\text{in}~#4,}
\newcommand{\BEq}[4]{#1&#2#3	 &\quad &\text{on}~#4}
\newcommand{\br}[1]{\left(#1\right)}



%Logical symbols
\newcommand{\Ra}{\Rightarrow}
\newcommand{\ra}{\rightarrow}
\newcommand{\hra}{\hookrightarrow}
\newcommand{\mt}{\mapsto}

% Aleksandrov Reflection Macros
\DeclareMathOperator{\reflectionvector}{V}
\DeclareMathOperator{\reflectionangle}{\delta}
\newcommand{\reflectionplane}[1][\reflectionvector]{\ensuremath{P_{#1}}}
\newcommand{\reflectionmap}[1][\reflectionvector]{\ensuremath{R_{#1}}}
\newcommand{\reflectionset}[2][\reflectionvector]{\ensuremath{{#2}_{#1}}}
\newcommand{\reflectionhalfspace}[1][\reflectionvector]{\ensuremath{\reflectionset[{#1}]{H}}}
\DeclareMathOperator{\vertvec}{e}
\DeclareMathOperator{\origin}{O}
\DeclareMathOperator{\radialprojection}{\pi}
\DeclareMathOperator{\height}{h}
\DeclareMathOperator{\equator}{E}
\newcommand{\ip}[2]{\ensuremath{\langle{#1},{#2}\rangle}}
\DeclareMathOperator{\intersect}{\cap}
\DeclareMathOperator{\union}{\cup}
\DeclareMathOperator{\nor}{\nu}
\DeclareMathOperator{\basepoint}{p_0}
\DeclareMathOperator{\radialdistance}{r}

%Fonts
\newcommand{\mc}{\mathcal}
\renewcommand{\it}{\textit}
\newcommand{\mrm}{\mathrm}

%Spacing
\newcommand{\hp}{\hphantom}


%\parindent 0 pt

\protected\def\ignorethis#1\endignorethis{}
\let\endignorethis\relax
\def\TOCstop{\addtocontents{toc}{\ignorethis}}
\def\TOCstart{\addtocontents{toc}{\endignorethis}}


\begin{document}

\title[Nonsense]
 {Nonsense On Pull-back Bundles and Connections}

\curraddr{}
\email{}
\date{\today}

\dedicatory{}
\subjclass[2010]{}
\keywords{}

\maketitle

\section{Pull Back Bundle}

Let \(\pi^E : E \to N\) be a smooth, rank \(k\), vector bundle. Given a smooth map \(f : M \to N\), there is a smooth vector bundle, the \emph{pull-back bundle} \(f^{\ast} \pi : f^{\ast}E \to M\) whose sections over an open set \(U  \subset M\) are
\[
\Gamma(U, f^{\ast} E) = \{s : U \to E | \pi^E \circ s = f\}.
\]

Given a section \(s \in \Gamma(V, E)\) of \(E\), we can pull it back to \(f^{\ast} s \in \Gamma(f^{-1}(V), f^{\ast} E)\) whose definition is,
\[
f^{\ast} (x) = s(f(x)), x \in f^{-1}(V).
\]

While it is not true that in general, a local section \(S\) of \(f^{\ast} E\) is equal to a pull-back section \(f^{\ast} s\), it is true that the sections of \(f^{\ast} E\) are \emph{locally generated} by pull-back sections. That is, any open set \(U \subset M\) is covered by sets \(U_{\alpha} = f^{-1}(V_{\alpha})\) where \(V_{\alpha}\) trivialise \(E\).

Then for each \(\alpha\), we have a local frame. \(\{s^{\alpha}_i\}_{i=1}^k\) for \(E\) and \(\{f^{\ast} s^{\alpha}_i\}\) generate \(\Gamma(U_{\alpha}, f^{\ast} E\) in that any section \(S \in \Gamma(U_{\alpha}, f^{\ast} E)\) may be written,
\[
S(x) = G^i(x) s^{\alpha}_i(x)
\]
for smooth functions \(G^i \in C^{\infty}(U_{\alpha}, \R)\). It's important to note here that in general, \(G^i \ne f^{\ast} g^i\) for smooth functions \(g^i\) (otherwise we would have \(S = f^{\ast} s\) where \(s = G^i s^{\alpha}_i\).

In the particular case that \(f\) is an immersion, any smooth function \(G : U \to \R\) may be locally extended to \(g : V \to \R\) in the sense that \(G = f^{\ast} g\) along the open set \(f^{-1}(V) \cap U\), but not in general globally. If \(f\) is an embedding, \(G = f^{\ast} g\) for \(g\) a smooth function on a neighbourhood of \(F(M)\). Thus, in the case of an immersion, all sections \(S \in \Gamma(U, f^{\ast} E)\) are locally pull-backs \(S = f^{\ast} s\). For an embedding, all sections \(S \in \Gamma(U, f^{\ast} E)\) are globally pull-backs \(S = f^{\ast} s\). Once more, let us emphasise the point that for a general smooth map, \(f: M \to N\) sections of the pull-back bundle cannot even be written locally as pull-back sections.

\begin{example}
A particular example is the tangent bundle, \(TN \to N\). In this case, we get a bundle map,
\begin{align*}
f_{\ast} : TM &\to f^{\ast} TN \\
X &\mapsto df \cdot X,
\end{align*}
that is,
\[
(f_{\ast} X) (x) = df_{x} \cdot X(x).
\]

As noted above, \(f^{\ast} X\) so defined will not in general be extendable, even locally to a vector field on \(N\), but it does define a section of the pull-back bundle. If \(f\) is an embedding, \(f_{\ast} X\) may extended globally to vector field on a neighbourhood of \(f(M)\), and if \(f\) is an immersion, such an extension is possible locally.

For immersions, by definition \(df_x\) is injective at each \(x\) and so we see that \(f_{\ast}\) is an injective as a bundle map and hence,
\[
f_{\ast} (TM) \hookrightarrow f^{\ast} TN
\]
realises \(TM\) as isomorphic to a sub-bundle of \(f^{\ast} TN\).

The normal bundle \(\nu M\) is defined as the quotient bundle \(f^{\ast} TN/\nu M\) which gives an exact sequence,
\[
0 \to TM \overset{f_{\ast}}{\to} f^{\ast} TN \to \nu M \to 0
\]
of vector bundles

If \(TN\) has a metric, then the metric splits the exact sequence. That is we have the orthogonal decomposition,
\[
f^{\ast} TN \simeq TM \oplus \nu M
\]
and we have an inclusion,
\[
\nu M \to f^{\ast} TN.
\]
\end{example}

\section{Connections on Pull-back Bundles}

Suppose now our vector bundle \(\pi^E : E \to N\) has a connection \(D\). There are a couple of equivalent ways to define the pull back connection \(f^{\ast} D\) on \(f^{\ast} E\). The defining property of the pull-back connection is,
\[
(f^{\ast} D)_X f^{\ast} s = D_{f_{\ast} X} s
\]
for any tangent vector \(X \in TM\). The point now is to show,
\begin{enumerate}
\item \(f^{\ast} D\) is well defined,
\item \(f^{\ast} D\) is the unique connection with the above property.
\end{enumerate}

\end{document}
